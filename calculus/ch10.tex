\section*{Chapter 10. Differential equations of first order}

\subsection*{Exercise 10.1}

Solve
\begin{equation*}
    \tan(x) \tan(y) dx + \sec^2(y) dy = 0.
\end{equation*}

\subsection*{Solution}

We can solve this differential equation by separating the variables.
We have
\begin{equation*}
    -\tan(x) dx = \frac{\sec^2(y)}{\tan(y)} dy.
\end{equation*}
For the left side integrate
\begin{equation*}
    \int -\tan(x) dx = \int \frac{-\sin(x) dx}{\cos(x)} = \int \frac{dt}{t} = \log(\cos(x)) + C_1.
\end{equation*}
For the right side integrate
\begin{equation*}
    \int \frac{\sec^2(y)}{\tan(y)} dy
        = \int \frac{1}{\sin(y)\cos(y)} dy
        = \int \frac{2}{\sin(2y)} dy
        = \int \frac{1 + t^2}{2t} \frac{2 dt}{1 + t^2}
        = \log(t) + C_1
        = \log(\tan(y)) + C_1,
\end{equation*}
where we used that $\sin(y)\cos(y) = \frac{1}{2}\sin(2y)$ and Weierstrass' substitution $t = \tan(y)$ such that $\sin(2y) = \frac{2t}{1 + t^2}$.
We have
\begin{equation*}
    \log(\tan(y)) + C_1 = \log(\cos(x)) + C_2, \quad \rightarrow \quad \frac{\tan(y)}{\cos(x)} = C.
\end{equation*}

Note that the right side can be integrated more easy.
We have
\begin{equation*}
    \frac{1}{\sin(y)\cos(y)}
        = \frac{\sin^2(y) + \cos^2(y)}{\sin(y)\cos(y)}
        = \frac{\sin(y)}{\cos(y)} + \frac{\cos(y)}{\sin(y)}.
\end{equation*}
The first part of the sum integrates to $-\log(\cos(y))$ and the second part to $\log(\sin(y))$.
So, together, we get $\log(\tan(y))$.


\subsection*{Exercise 10.2}

Solve
\begin{equation*}
    (1 + x^2) y dx + (1 - y^2) x dy = 0.
\end{equation*}

\subsection*{Solution}

We have
\begin{equation*}
    0 = \left(\frac{1}{x} + x\right)dx + \left(\frac{1}{y} - y\right)dy,
\end{equation*}
so
\begin{equation*}
    C = \log(xy) + \frac{1}{2}x^2 + \frac{1}{2}y^2.
\end{equation*}


\subsection*{Exercise 10.3}

Solve
\begin{equation*}
    xy(1 + x^2) dy - (1 - y^2) dy = 0.
\end{equation*}

\subsection*{Solution}

Rearranging the ordinary differential equation, we have
\begin{equation*}
    \frac{y}{1 - y^2}dy - \frac{1}{x} dx + \frac{x}{1 + x^2} dx = 0.
\end{equation*}
Integrate on both sides gives
\begin{equation*}
    -\frac{1}{2} \log(1 - y^2) - \log(x) + \frac{1}{2} \log(1 + x^2) = C.
\end{equation*}
Finally, taking the exponential, we find the solution
\begin{equation*}
    1 + x^2 = C x^2 (1 - y^2).
\end{equation*}


\subsection*{Exercise 10.4}

Solve
\begin{equation*}
    (x + y)dx + xdy = 0.
\end{equation*}

\subsection*{Solution}

This differential equation is homogenous of order 1.
Take $y = \lambda x$ such that $dy = \lambda dx + x d\lambda$.
The differential equation becomes
\begin{equation*}
    x(1 + 2\lambda) dx + x^2 d\lambda = 0, \quad \rightarrow \quad \frac{1}{x} dx + \frac{1}{1 + 2\lambda} d\lambda = 0.
\end{equation*}
Integrating on both sides gives
\begin{equation*}
    \log(x) + \frac{1}{2} \log(1 + 2\lambda) = \log(x \sqrt{1 + 2\lambda}) = C.
\end{equation*}
Taking the exponential on both sides and replacing $\lambda = \frac{y}{x}$ gives the concise form
\begin{equation*}
    x^2 + 2xy = C.
\end{equation*}


\subsection*{Exercise 10.5}

Solve
\begin{equation*}
    (y - \sqrt{x^2 + y^2})dx - xdy = 0.
\end{equation*}

\subsection*{Solution}

Note that this differential equation is a homogeneous equation of order 1.
Replace $y = \lambda x$, such that $dy = xd\lambda + \lambda dx$.
We have
\begin{equation*}
    -x\sqrt{1 + \lambda^2} dx - x^2 d\lambda = 0, \quad \rightarrow \quad \frac{1}{x} dx + \frac{1}{\sqrt{1 + \lambda^2}} d\lambda = 0.
\end{equation*}
Integrate on both sides and recognize that the $\lambda$ solution is $\sinh^{-1}(\lambda) = \log(\lambda + \sqrt{1 + \lambda^2})$.
We get
\begin{equation*}
    \log(x) + \log(\lambda + \sqrt{1 + \lambda^2}) = \log(x(\lambda + \sqrt{1 + \lambda^2})) = C.
\end{equation*}
Replace $\lambda = y/x$ and take the exponential gives
\begin{equation*}
    y + \sqrt{x^2 + y^2} = C.
\end{equation*}

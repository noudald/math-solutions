\section*{Chapter 6. The Definite Integral}

\subsection*{Exercise 6.1}

If $f(x)$ is an odd function, that is if $f(-x) = -f(x)$, prove that
\begin{equation*}
    \int_{-a}^a f(x) dx = 0.
\end{equation*}

\subsection*{Solution}
\begin{equation*}
    \begin{split}
        \int_{-a}^a f(x) dx
            &= \int_{-a}^0 f(x) dx + \int_0^a f(x) dx \\
            &= \int_{-a}^0 -f(-x) dx + \int_0^a f(x) dx \\
            &= \int_a^0 f(y) dy + \int_0^a f(x) dx \\
            &= -\int_0^a f(y) dy + \int_0^a f(x) dx
            = 0,
    \end{split}
\end{equation*}
where we made the substitution $y = -x$ in the first integral of the third equality.


\subsection*{Exercise 6.2}

If $f(x)$ is an even function, that is, if $f(-x) = f(x)$, prove that
\begin{equation*}
    \int_{-a}^a f(x) dx = 2 \int_0^a f(x) dx.
\end{equation*}

\subsection*{Solution}

\begin{equation*}
    \int_{-a}^a f(x) dx
        = \int_{-a}^0 f(x) dx + \int_{0}^{a} f(x)dx
        = -\int_a^0 f(y)dy + \int_0^a f(x)dx
        = 2\int_0^a f(x)dx,
\end{equation*}
where we made the substitution $y = -x$ in the first integral of the second equality.


\subsection*{Exercise 6.3}

If $f(a - x) = f(x)$, prove that
\begin{equation*}
    \int_0^a f(x) dx = 2 \int_0^{\frac{a}{2}} f(x) dx.
\end{equation*}

\subsection*{Solution}

\begin{equation*}
    \begin{split}
        \int_0^a f(x) dx
            &= \int_0^{\frac{a}{2}} f(x) dx + \int_{\frac{a}{2}}^a f(x) dx \\
            &= \int_0^{\frac{a}{2}} f(x) dx + \int_{\frac{a}{2}}^a f(a - x) dx \\
            &= \int_0^{\frac{a}{2}} f(x) dx - \int_{-\frac{a}{2}}^{0} f(y) dy
            = 2 \int_0^{\frac{a}{2}} f(x) dx,
    \end{split}
\end{equation*}
where we made the substitution $y = a - x$ in the second integral of the second equality.


\subsection*{Exercise 6.4}

Show that
\begin{equation*}
    \int_0^{2k\pi} f(sin(x)) dx = k \int_0^{2\pi} f(sin(x)) dx.
\end{equation*}

\subsection*{Solution}

This is a special case of Exercise 6.5.
Note that if $g(x) = f(sin(x))$, then $g(x + 2\pi) = g(x)$.
By Exercise 6.5, we get the result.


\subsection*{Exercise 6.5}

If $f(x)$ has a period $a$, that is, if $f(x + a) = f(x)$, prove that
\begin{equation*}
    \int_0^{ka} f(x) dx = k \int_0^a f(x) dx,
\end{equation*}
for any integer $k$.

\subsection*{Solution}

Note that for any integer $\ell$ we have
\begin{equation*}
    \int_{\ell a}^{(\ell + 1)a} f(x) dx = \int_0^a f(y + \ell a) dy = \int_0^a f(y) dy,
\end{equation*}
where we made the substitution $y = x - \ell a$.
Therefore,
\begin{equation*}
    \int_0^{ka} f(x) dx
        = \int_0^a f(x) dx + \int_a^{2a} f(x) dx + ... + \int_{(k-1)a}^{ka} f(x) dx
        = k \int_0^a f(x) dx.
\end{equation*}


\subsection*{Exercise 6.6}

If $a < b$, and $f_1(x) < f_2(x) < f_3(x)$ for any $x$ in the interval $(a, b)$, prove that
\begin{equation*}
    \int_a^b f_1(x)dx < \int_a^b f_2(x) < \int_a^b f_3(x).
\end{equation*}

\subsection*{Solution}

It's sufficient to show the result for $f_1(x)$ and $f_2(x)$, then replace $f_1$ by $f_2$ and $f_2$ by $f_3$.
Let $g(x) = f_2(x) - f_1(x)$.
Note that $g(x) > 0$ for all $x$ in $(a, b)$.
By 56.2 and 56.5 we have
\begin{equation*}
    \int_a^b f_2(x) dx - \int_a^b f_1(x) = \int_a^b g(x) dx = (b - a) g(\xi) > 0,
\end{equation*}
for some $\xi \in (a, b)$.
Therefore
\begin{equation*}
    \int_a^b f_1(x) dx < \int_a^b f_2(x) dx.
\end{equation*}


\subsection*{Exercise 6.7}

If $m$ and $M$ are the smallest and largest values of $f(x)$ in the interval $(a, b)$, and $\phi(x) > 0$ in the interval, prove that
\begin{equation*}
    m \int_a^b \phi(x) dx < \int_a^b f(x) \phi(x) dx < M \int_a^b \phi(x) dx,
\end{equation*}
and therefore
\begin{equation*}
    \int_a^b f(x) \phi(x) dx = f(\xi) \int_a^b \phi(x) dx,
\end{equation*}
for some $\xi \in (a, b)$.

\subsection*{Solution}

For the first part of the exercise take $f_1(x) = m \phi(x)$, $f_2(x) = f(x) \phi(x)$, and $f_3(x) = M \phi(x)$.
Then $f_1(x) < f_2(x) < f_3(x)$, and we can apply Exercise 6.6 to get the result.
Define the function
\begin{equation*}
    g(y) = f(y) \int_a^b \phi(x) dx.
\end{equation*}
By the previous result $g(y)$ takes any value between $m \int_a^b \phi(x) dx$ and $M \int_a^b \phi(x) dx$.
In particular, as $g(y)$ is continuous, there is a $\xi \in (a, b)$ such that
\begin{equation*}
    \int_a^b f(x) \phi(x) dx = g(\xi) = f(\xi) \int_a^b \phi(x) dx.
\end{equation*}


\subsection*{Exercise 6.8}

Evaluate
\begin{equation*}
    \int_0^3 (1 + x^2)^{\frac{3}{2}} dx,
\end{equation*}
by Simpson's rule, taking $n = 3$.

\subsection*{Solution}
By Simpson's rule, using $n = 3$,
\begin{equation*}
    \int_0^3 (1 + x^2)^{\frac{3}{2}} dx
        \approx \frac{3}{3 \cdot 6} \left(f(0) + 4f\left(\frac{1}{2}\right) + 2f(1) + 4f\left(\frac{3}{2}\right) + 2f(2) + 4f\left(\frac{5}{2}\right) + f(3)\right)
        = 27.95857559...
\end{equation*}

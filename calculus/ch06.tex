\section*{Chapter 6. The Definite Integral}

\subsection*{Exercise 6.1}

If $f(x)$ is an odd function, that is if $f(-x) = -f(x)$, prove that
\begin{equation*}
    \int_{-a}^a f(x) dx = 0.
\end{equation*}

\subsection*{Solution}
\begin{equation*}
    \begin{split}
        \int_{-a}^a f(x) dx
            &= \int_{-a}^0 f(x) dx + \int_0^a f(x) dx \\
            &= \int_{-a}^0 -f(-x) dx + \int_0^a f(x) dx \\
            &= \int_a^0 f(y) dy + \int_0^a f(x) dx \\
            &= -\int_0^a f(y) dy + \int_0^a f(x) dx
            = 0,
    \end{split}
\end{equation*}
where we made the substitution $y = -x$ in the first integral of the third equality.


\subsection*{Exercise 6.2}

If $f(x)$ is an even function, that is, if $f(-x) = f(x)$, prove that
\begin{equation*}
    \int_{-a}^a f(x) dx = 2 \int_0^a f(x) dx.
\end{equation*}

\subsection*{Solution}

\begin{equation*}
    \int_{-a}^a f(x) dx
        = \int_{-a}^0 f(x) dx + \int_{0}^{a} f(x)dx
        = -\int_a^0 f(y)dy + \int_0^a f(x)dx
        = 2\int_0^a f(x)dx,
\end{equation*}
where we made the substitution $y = -x$ in the first integral of the second equality.


\subsection*{Exercise 6.3}

If $f(a - x) = f(x)$, prove that
\begin{equation*}
    \int_0^a f(x) dx = 2 \int_0^{\frac{a}{2}} f(x) dx.
\end{equation*}

\subsection*{Solution}

\begin{equation*}
    \begin{split}
        \int_0^a f(x) dx
            &= \int_0^{\frac{a}{2}} f(x) dx + \int_{\frac{a}{2}}^a f(x) dx \\
            &= \int_0^{\frac{a}{2}} f(x) dx + \int_{\frac{a}{2}}^a f(a - x) dx \\
            &= \int_0^{\frac{a}{2}} f(x) dx - \int_{-\frac{a}{2}}^{0} f(y) dy
            = 2 \int_0^{\frac{a}{2}} f(x) dx,
    \end{split}
\end{equation*}
where we made the substitution $y = a - x$ in the second integral of the second equality.


\subsection*{Exercise 6.4}

Show that
\begin{equation*}
    \int_0^{2k\pi} f(sin(x)) dx = k \int_0^{2\pi} f(sin(x)) dx.
\end{equation*}

\subsection*{Solution}

This is a special case of Exercise 6.5.
Note that if $g(x) = f(sin(x))$, then $g(x + 2\pi) = g(x)$.
By Exercise 6.5, we get the result.


\subsection*{Exercise 6.5}

If $f(x)$ has a period $a$, that is, if $f(x + a) = f(x)$, prove that
\begin{equation*}
    \int_0^{ka} f(x) dx = k \int_0^a f(x) dx,
\end{equation*}
for any integer $k$.

\subsection*{Solution}

Note that for any integer $\ell$ we have
\begin{equation*}
    \int_{\ell a}^{(\ell + 1)a} f(x) dx = \int_0^a f(y + \ell a) dy = \int_0^a f(y) dy,
\end{equation*}
where we made the substitution $y = x - \ell a$.
Therefore,
\begin{equation*}
    \int_0^{ka} f(x) dx
        = \int_0^a f(x) dx + \int_a^{2a} f(x) dx + ... + \int_{(k-1)a}^{ka} f(x) dx
        = k \int_0^a f(x) dx.
\end{equation*}


\subsection*{Exercise 6.6}

If $a < b$, and $f_1(x) < f_2(x) < f_3(x)$ for any $x$ in the interval $(a, b)$, prove that
\begin{equation*}
    \int_a^b f_1(x)dx < \int_a^b f_2(x) < \int_a^b f_3(x).
\end{equation*}

\subsection*{Solution}

It's sufficient to show the result for $f_1(x)$ and $f_2(x)$, then replace $f_1$ by $f_2$ and $f_2$ by $f_3$.
Let $g(x) = f_2(x) - f_1(x)$.
Note that $g(x) > 0$ for all $x$ in $(a, b)$.
By 56.2 and 56.5 we have
\begin{equation*}
    \int_a^b f_2(x) dx - \int_a^b f_1(x) = \int_a^b g(x) dx = (b - a) g(\xi) > 0,
\end{equation*}
for some $\xi \in (a, b)$.
Therefore
\begin{equation*}
    \int_a^b f_1(x) dx < \int_a^b f_2(x) dx.
\end{equation*}


\subsection*{Exercise 6.7}

If $m$ and $M$ are the smallest and largest values of $f(x)$ in the interval $(a, b)$, and $\phi(x) > 0$ in the interval, prove that
\begin{equation*}
    m \int_a^b \phi(x) dx < \int_a^b f(x) \phi(x) dx < M \int_a^b \phi(x) dx,
\end{equation*}
and therefore
\begin{equation*}
    \int_a^b f(x) \phi(x) dx = f(\xi) \int_a^b \phi(x) dx,
\end{equation*}
for some $\xi \in (a, b)$.

\subsection*{Solution}

For the first part of the exercise take $f_1(x) = m \phi(x)$, $f_2(x) = f(x) \phi(x)$, and $f_3(x) = M \phi(x)$.
Then $f_1(x) < f_2(x) < f_3(x)$, and we can apply Exercise 6.6 to get the result.
Define the function
\begin{equation*}
    g(y) = f(y) \int_a^b \phi(x) dx.
\end{equation*}
By the previous result $g(y)$ takes any value between $m \int_a^b \phi(x) dx$ and $M \int_a^b \phi(x) dx$.
In particular, as $g(y)$ is continuous, there is a $\xi \in (a, b)$ such that
\begin{equation*}
    \int_a^b f(x) \phi(x) dx = g(\xi) = f(\xi) \int_a^b \phi(x) dx.
\end{equation*}


\subsection*{Exercise 6.8}

Evaluate
\begin{equation*}
    \int_0^3 (1 + x^2)^{\frac{3}{2}} dx,
\end{equation*}
by Simpson's rule, taking $n = 3$.

\subsection*{Solution}
By Simpson's rule, using $n = 3$,
\begin{equation*}
    \int_0^3 (1 + x^2)^{\frac{3}{2}} dx
        \approx \frac{3}{3 \cdot 6} \left(f(0) + 4f\left(\frac{1}{2}\right) + 2f(1) + 4f\left(\frac{3}{2}\right) + 2f(2) + 4f\left(\frac{5}{2}\right) + f(3)\right)
        = 27.95857559...
\end{equation*}


\subsection*{Exercise 6.18}

Evaluate
\begin{equation*}
    I(\alpha) = \int_0^1 \frac{x^{\alpha} - 1}{\log(x)} dx.
\end{equation*}

\subsection*{Solution}

We have initial condition $I(\alpha = 0) = 0$.
Differentiate under the integral with respect to $\alpha$,
\begin{equation*}
    I'(\alpha) = \int_0^1 \frac{d}{d\alpha} \left( \frac{x^{\alpha} - 1}{\log(x)} \right) dx
        = \int_0^1 x^{\alpha} d\alpha
        = \left. \frac{1}{1 + \alpha} x^{\alpha+1} \right|_0^1
        = \frac{1}{1 + \alpha}.
\end{equation*}
Therefore, $I(\alpha) = \log(1 + \alpha) + C$.
Using initial condition $I(\alpha = 0) = 0$ we have $C = 0$.
Hence $I(\alpha) = \log(1 + \alpha)$.


\subsection*{Exercise 6.19}

By successive differentiations of $\int_0^1 x^n dx = \frac{1}{n + 1}$ obtain
\begin{equation*}
    \int_0^1 x^n \log(x)^m dx = (-1)^m \frac{m!}{(n + 1)^{m + 1}}.
\end{equation*}

\subsection*{Solution}

Let $I(n) = \int_0^1 x^n dx = \frac{1}{n + 1}$.
As $n \geq 1$, $I(n)$ is a continuous differentiable function, and we can differentiate the function $m$ times with respect to $n$.
On the one hand we have
\begin{equation*}
    \frac{d^m}{dn^m} I(n) = \int_0^1 \frac{d^m}{dn^m} x^n dx = \int_0^1 x^n \log(x)^m dx,
\end{equation*}
while on the other hand
\begin{equation*}
    \frac{d^m}{dn^m} I(n) = \frac{d^m}{dn^m} \frac{1}{n + 1} = (-1)^m \frac{m!}{(n + 1)^{m + 1}}.
\end{equation*}
This gives the result.


\subsection*{Exercise 6.20}

From
\begin{equation*}
    \int_0^{\pi} \frac{dx}{a - \cos(x)} = \frac{\pi}{\sqrt{a^2 - 1}}, \quad a > 1,
\end{equation*}
find
\begin{equation*}
    \int_0^\pi \log\left(\frac{b - \cos(x)}{a - \cos(x)} \right) dx = \pi \log\left( \frac{b + \sqrt{b^2 - 1}}{a + \sqrt{a^2 - 1}} \right).
\end{equation*}

\subsection*{Solution}
Define
\begin{equation*}
    I(a) = \int_0^\pi \log(a - \cos(x)) dx.
\end{equation*}
From the multiplication law of the logarithm we have
\begin{equation*}
    \int_0^\pi \log\left(\frac{b - \cos(x)}{a - \cos(x)} \right) dx
        = \int_0^\pi \log(b - \cos(x)) dx - \int_0^\pi \log(a - \cos(x)) dx
        = I(b) - I(a).
\end{equation*}
We differentiate under the integral to get
\begin{equation*}
    I'(a) = \int_0^{\pi} \frac{dx}{a - \cos(x)}.
\end{equation*}
To solve this integral apply Weierstrass half tangent substitution $t = \tan(\frac{x}{2})$, such that $dx = \frac{2}{1 + t^2}dt$ and $\cos(x) = \frac{1 - t^2}{1 + t^2}$.
We have
\begin{equation*}
    \begin{split}
        \int_0^\pi \frac{dx}{a - \cos(x)}
            &= \int_0^\infty \frac{1}{a - \frac{1 - t^2}{1 + t^2}} \frac{2}{1 + t^2} dt \\
            &= \int_0^\infty \frac{2}{(a - 1) + (a + 1)t^2} dt \\
            &= \frac{2}{a - 1} \int_0^{\infty} \frac{1}{1 + \left(\sqrt{\frac{a - 1}{a + 1}}t\right)^2} dt \\
            &= \frac{2}{\sqrt{(a - 1)(a + 1)}} \int_0^\infty \frac{1}{1 + s^2} ds \\
            &= \left. \frac{2}{\sqrt{a^2 - 1}} \arctan(s) \right|_0^{\infty}
            = \frac{\pi}{\sqrt{a^2 - 1}}.
    \end{split}
\end{equation*}
Finally, note that
\begin{equation*}
    \frac{d}{da} \log\left(a + \sqrt{a^2 - 1}\right)
        = \frac{1}{a + \sqrt{a^2 - 1}} \left(1 + \frac{a}{\sqrt{a^2 - 1}}\right)
        = \frac{1}{a + \sqrt{a^2 - 1}} \left(\frac{a + a\sqrt{a^2 - 1}}{\sqrt{a^2 - 1}}\right)
        = \frac{1}{\sqrt{a^2 - 1}}.
\end{equation*}
Alternatively, instead of deus ex machina this integral, we calculate the integral directly using the substitution $a = \cosh(t)$, such that $da = \sinh(t) dt$.
\begin{equation*}
    \int \frac{da}{\sqrt{a^2 - 1}}
        = \int \frac{\sinh(t)}{\sqrt{\cosh^2(t) - 1}} dt
        = t.
\end{equation*}
Note that
\begin{equation*}
    a = \frac{e^t + e^{-t}}{2}
        \quad \leftrightarrow \quad e^t = a \pm \sqrt{a^2 - 1}
        \quad \leftrightarrow \quad t = \log(a \pm \sqrt{a^2 - 1}).
\end{equation*}
Taking the correct sign in the logarithm gives the result.
So
\begin{equation*}
    I(a) = \int \frac{\pi}{\sqrt{a^2 - 1}} da = \pi \log\left(a + \sqrt{a^2 - 1}\right) + C,
\end{equation*}
such that
\begin{equation*}
    \int_0^\pi \log\left(\frac{b - \cos(x)}{a - \cos(x)} \right) dx
        = I(b) - I(a)
        = \pi \log\left( \frac{b + \sqrt{b^2 - 1}}{a + \sqrt{a^2 - 1}} \right).
\end{equation*}

\section*{Chapter 6. The Definite Integral}

\subsection*{Exercise 6.1}

If $f(x)$ is an odd function, that is if $f(-x) = -f(x)$, prove that
\begin{equation*}
    \int_{-a}^a f(x) dx = 0.
\end{equation*}

\subsection*{Solution}
\begin{equation*}
    \begin{split}
        \int_{-a}^a f(x) dx
            &= \int_{-a}^0 f(x) dx + \int_0^a f(x) dx \\
            &= \int_{-a}^0 -f(-x) dx + \int_0^a f(x) dx \\
            &= \int_a^0 f(y) dy + \int_0^a f(x) dx \\
            &= -\int_0^a f(y) dy + \int_0^a f(x) dx
            = 0,
    \end{split}
\end{equation*}
where we made the substitution $y = -x$ in the first integral of the third equality.


\subsection*{Exercise 6.2}

If $f(x)$ is an even function, that is, if $f(-x) = f(x)$, prove that
\begin{equation*}
    \int_{-a}^a f(x) dx = 2 \int_0^a f(x) dx.
\end{equation*}

\subsection*{Solution}

\begin{equation*}
    \int_{-a}^a f(x) dx
        = \int_{-a}^0 f(x) dx + \int_{0}^{a} f(x)dx
        = -\int_a^0 f(y)dy + \int_0^a f(x)dx
        = 2\int_0^a f(x)dx,
\end{equation*}
where we made the substitution $y = -x$ in the first integral of the second equality.


\subsection*{Exercise 6.3}

If $f(a - x) = f(x)$, prove that
\begin{equation*}
    \int_0^a f(x) dx = 2 \int_0^{\frac{a}{2}} f(x) dx.
\end{equation*}

\subsection*{Solution}

\begin{equation*}
    \begin{split}
        \int_0^a f(x) dx
            &= \int_0^{\frac{a}{2}} f(x) dx + \int_{\frac{a}{2}}^a f(x) dx \\
            &= \int_0^{\frac{a}{2}} f(x) dx + \int_{\frac{a}{2}}^a f(a - x) dx \\
            &= \int_0^{\frac{a}{2}} f(x) dx - \int_{-\frac{a}{2}}^{0} f(y) dy
            = 2 \int_0^{\frac{a}{2}} f(x) dx,
    \end{split}
\end{equation*}
where we made the substitution $y = a - x$ in the second integral of the second equality.


\subsection*{Exercise 6.4}

Show that
\begin{equation*}
    \int_0^{2k\pi} f(sin(x)) dx = k \int_0^{2\pi} f(sin(x)) dx.
\end{equation*}

\subsection*{Solution}

This is a special case of Exercise 6.5.
Note that if $g(x) = f(sin(x))$, then $g(x + 2\pi) = g(x)$.
By Exercise 6.5, we get the result.


\subsection*{Exercise 6.5}

If $f(x)$ has a period $a$, that is, if $f(x + a) = f(x)$, prove that
\begin{equation*}
    \int_0^{ka} f(x) dx = k \int_0^a f(x) dx,
\end{equation*}
for any integer $k$.

\subsection*{Solution}

Note that for any integer $\ell$ we have
\begin{equation*}
    \int_{\ell a}^{(\ell + 1)a} f(x) dx = \int_0^a f(y + \ell a) dy = \int_0^a f(y) dy,
\end{equation*}
where we made the substitution $y = x - \ell a$.
Therefore,
\begin{equation*}
    \int_0^{ka} f(x) dx
        = \int_0^a f(x) dx + \int_a^{2a} f(x) dx + ... + \int_{(k-1)a}^{ka} f(x) dx
        = k \int_0^a f(x) dx.
\end{equation*}


\subsection*{Exercise 6.6}

If $a < b$, and $f_1(x) < f_2(x) < f_3(x)$ for any $x$ in the interval $(a, b)$, prove that
\begin{equation*}
    \int_a^b f_1(x)dx < \int_a^b f_2(x) < \int_a^b f_3(x).
\end{equation*}

\subsection*{Solution}

It's sufficient to show the result for $f_1(x)$ and $f_2(x)$, then replace $f_1$ by $f_2$ and $f_2$ by $f_3$.
Let $g(x) = f_2(x) - f_1(x)$.
Note that $g(x) > 0$ for all $x$ in $(a, b)$.
By 56.2 and 56.5 we have
\begin{equation*}
    \int_a^b f_2(x) dx - \int_a^b f_1(x) = \int_a^b g(x) dx = (b - a) g(\xi) > 0,
\end{equation*}
for some $\xi \in (a, b)$.
Therefore
\begin{equation*}
    \int_a^b f_1(x) dx < \int_a^b f_2(x) dx.
\end{equation*}


\subsection*{Exercise 6.7}

If $m$ and $M$ are the smallest and largest values of $f(x)$ in the interval $(a, b)$, and $\phi(x) > 0$ in the interval, prove that
\begin{equation*}
    m \int_a^b \phi(x) dx < \int_a^b f(x) \phi(x) dx < M \int_a^b \phi(x) dx,
\end{equation*}
and therefore
\begin{equation*}
    \int_a^b f(x) \phi(x) dx = f(\xi) \int_a^b \phi(x) dx,
\end{equation*}
for some $\xi \in (a, b)$.

\subsection*{Solution}

For the first part of the exercise take $f_1(x) = m \phi(x)$, $f_2(x) = f(x) \phi(x)$, and $f_3(x) = M \phi(x)$.
Then $f_1(x) < f_2(x) < f_3(x)$, and we can apply Exercise 6.6 to get the result.
Define the function
\begin{equation*}
    g(y) = f(y) \int_a^b \phi(x) dx.
\end{equation*}
By the previous result $g(y)$ takes any value between $m \int_a^b \phi(x) dx$ and $M \int_a^b \phi(x) dx$.
In particular, as $g(y)$ is continuous, there is a $\xi \in (a, b)$ such that
\begin{equation*}
    \int_a^b f(x) \phi(x) dx = g(\xi) = f(\xi) \int_a^b \phi(x) dx.
\end{equation*}


\subsection*{Exercise 6.8}

Evaluate
\begin{equation*}
    \int_0^3 (1 + x^2)^{\frac{3}{2}} dx,
\end{equation*}
by Simpson's rule, taking $n = 3$.

\subsection*{Solution}
By Simpson's rule, using $n = 3$,
\begin{equation*}
    \int_0^3 (1 + x^2)^{\frac{3}{2}} dx
        \approx \frac{3}{3 \cdot 6} \left(f(0) + 4f\left(\frac{1}{2}\right) + 2f(1) + 4f\left(\frac{3}{2}\right) + 2f(2) + 4f\left(\frac{5}{2}\right) + f(3)\right)
        = 27.95857559...
\end{equation*}


\subsection*{Exercise 6.18}

Evaluate
\begin{equation*}
    I(\alpha) = \int_0^1 \frac{x^{\alpha} - 1}{\log(x)} dx.
\end{equation*}

\subsection*{Solution}

We have initial condition $I(\alpha = 0) = 0$.
Differentiate under the integral with respect to $\alpha$,
\begin{equation*}
    I'(\alpha) = \int_0^1 \frac{d}{d\alpha} \left( \frac{x^{\alpha} - 1}{\log(x)} \right) dx
        = \int_0^1 x^{\alpha} d\alpha
        = \left. \frac{1}{1 + \alpha} x^{\alpha+1} \right|_0^1
        = \frac{1}{1 + \alpha}.
\end{equation*}
Therefore, $I(\alpha) = \log(1 + \alpha) + C$.
Using initial condition $I(\alpha = 0) = 0$ we have $C = 0$.
Hence $I(\alpha) = \log(1 + \alpha)$.


\subsection*{Exercise 6.19}

By successive differentiations of $\int_0^1 x^n dx = \frac{1}{n + 1}$ obtain
\begin{equation*}
    \int_0^1 x^n \log(x)^m dx = (-1)^m \frac{m!}{(n + 1)^{m + 1}}.
\end{equation*}

\subsection*{Solution}

Let $I(n) = \int_0^1 x^n dx = \frac{1}{n + 1}$.
As $n \geq 1$, $I(n)$ is a continuous differentiable function, and we can differentiate the function $m$ times with respect to $n$.
On the one hand we have
\begin{equation*}
    \frac{d^m}{dn^m} I(n) = \int_0^1 \frac{d^m}{dn^m} x^n dx = \int_0^1 x^n \log(x)^m dx,
\end{equation*}
while on the other hand
\begin{equation*}
    \frac{d^m}{dn^m} I(n) = \frac{d^m}{dn^m} \frac{1}{n + 1} = (-1)^m \frac{m!}{(n + 1)^{m + 1}}.
\end{equation*}
This gives the result.


\subsection*{Exercise 6.20}

From
\begin{equation*}
    \int_0^{\pi} \frac{dx}{a - \cos(x)} = \frac{\pi}{\sqrt{a^2 - 1}}, \quad a > 1,
\end{equation*}
find
\begin{equation*}
    \int_0^\pi \log\left(\frac{b - \cos(x)}{a - \cos(x)} \right) dx = \pi \log\left( \frac{b + \sqrt{b^2 - 1}}{a + \sqrt{a^2 - 1}} \right).
\end{equation*}

\subsection*{Solution}
Define
\begin{equation*}
    I(a) = \int_0^\pi \log(a - \cos(x)) dx.
\end{equation*}
From the multiplication law of the logarithm we have
\begin{equation*}
    \int_0^\pi \log\left(\frac{b - \cos(x)}{a - \cos(x)} \right) dx
        = \int_0^\pi \log(b - \cos(x)) dx - \int_0^\pi \log(a - \cos(x)) dx
        = I(b) - I(a).
\end{equation*}
We differentiate under the integral to get
\begin{equation*}
    I'(a) = \int_0^{\pi} \frac{dx}{a - \cos(x)}.
\end{equation*}
To solve this integral apply Weierstrass half tangent substitution $t = \tan(\frac{x}{2})$, such that $dx = \frac{2}{1 + t^2}dt$ and $\cos(x) = \frac{1 - t^2}{1 + t^2}$.
We have
\begin{equation*}
    \begin{split}
        \int_0^\pi \frac{dx}{a - \cos(x)}
            &= \int_0^\infty \frac{1}{a - \frac{1 - t^2}{1 + t^2}} \frac{2}{1 + t^2} dt \\
            &= \int_0^\infty \frac{2}{(a - 1) + (a + 1)t^2} dt \\
            &= \frac{2}{a - 1} \int_0^{\infty} \frac{1}{1 + \left(\sqrt{\frac{a - 1}{a + 1}}t\right)^2} dt \\
            &= \frac{2}{\sqrt{(a - 1)(a + 1)}} \int_0^\infty \frac{1}{1 + s^2} ds \\
            &= \left. \frac{2}{\sqrt{a^2 - 1}} \arctan(s) \right|_0^{\infty}
            = \frac{\pi}{\sqrt{a^2 - 1}}.
    \end{split}
\end{equation*}
Finally, note that
\begin{equation*}
    \frac{d}{da} \log\left(a + \sqrt{a^2 - 1}\right)
        = \frac{1}{a + \sqrt{a^2 - 1}} \left(1 + \frac{a}{\sqrt{a^2 - 1}}\right)
        = \frac{1}{a + \sqrt{a^2 - 1}} \left(\frac{a + a\sqrt{a^2 - 1}}{\sqrt{a^2 - 1}}\right)
        = \frac{1}{\sqrt{a^2 - 1}}.
\end{equation*}
Alternatively, instead of deus ex machina this integral, we calculate the integral directly using the substitution $a = \cosh(t)$, such that $da = \sinh(t) dt$.
\begin{equation*}
    \int \frac{da}{\sqrt{a^2 - 1}}
        = \int \frac{\sinh(t)}{\sqrt{\cosh^2(t) - 1}} dt
        = t.
\end{equation*}
Note that
\begin{equation*}
    a = \frac{e^t + e^{-t}}{2}
        \quad \leftrightarrow \quad e^t = a \pm \sqrt{a^2 - 1}
        \quad \leftrightarrow \quad t = \log(a \pm \sqrt{a^2 - 1}).
\end{equation*}
Taking the correct sign in the logarithm gives the result.
So
\begin{equation*}
    I(a) = \int \frac{\pi}{\sqrt{a^2 - 1}} da = \pi \log\left(a + \sqrt{a^2 - 1}\right) + C,
\end{equation*}
such that
\begin{equation*}
    \int_0^\pi \log\left(\frac{b - \cos(x)}{a - \cos(x)} \right) dx
        = I(b) - I(a)
        = \pi \log\left( \frac{b + \sqrt{b^2 - 1}}{a + \sqrt{a^2 - 1}} \right).
\end{equation*}


\subsection*{Exercise 6.21}

Test the convergence of the following integral
\begin{equation*}
    \int_1^{\infty} \frac{dx}{x\sqrt{1 + x^2}}.
\end{equation*}

\subsection*{Solution}

Take
\begin{equation*}
    \phi(x) = \frac{x^2}{\sqrt{1 + x^2}},
\end{equation*}
such that
\begin{equation*}
    \int_1^{\infty} \frac{dx}{x\sqrt{1 + x^2}} = \int_1^{\infty} \frac{\phi(x)}{x^2} dx.
\end{equation*}
As $x$ tends to infinity $\phi(x) \to 1$.
So $\phi(x)$ is bounded by a finite number for sufficient large values of $x$.
We can now apply Theorem 62.II to show that the integral converges.


\subsection*{Exercise 6.22}

Test the convergence of the following integral
\begin{equation*}
    \int_0^{\infty} \frac{\sin^2(x)}{x^2} dx.
\end{equation*}

\subsection*{Solution}

Take $\phi(x) = \sin^2(x)$.
Note that $0 \leq \phi(x) \leq 1$ and apply Theorem 62.II to show that this integral converges.


\subsection*{Exercise 6.23}

Test the convergence of the following integral
\begin{equation*}
    \int_a^{\infty} \frac{x^4}{(x^2 + a^2)^{\frac{5}{2}}} dx
\end{equation*}

\subsection*{Solution}

We have
\begin{equation*}
    \int_a^{\infty} \frac{x^4}{(x^2 + a^2)^{\frac{5}{2}}} dx
        = \int_a^{\infty} \frac{1}{x} \phi(x) dx,
\end{equation*}
where
\begin{equation*}
    \phi(x) = \frac{x^5}{(x^2 + a^2)^{\frac{5}{2}}}.
\end{equation*}
Note that $\phi(x) \to 1$ as $x \to \infty$.
By Theorem 62.II we see that the integral doesn't converge.


\subsection*{Exercise 6.24}

Test the convergence of the following integral
\begin{equation*}
    \int_0^{\infty} e^{-a^2 x^2} \cos(b x) dx.
\end{equation*}

\subsection*{Solution}

Take $\phi(x) = x^2 \cos(bx) e^{-a^2 x^2}$, such that
\begin{equation*}
    \int_0^{\infty} e^{-a^2 x^2} \cos(b x) dx = \int_0^{\infty} \frac{\phi(x)}{x^2} dx.
\end{equation*}
Note that
\begin{equation*}
    |\phi(x)| = \left| \frac{x^2 \cos(bx)}{e^{a^2 x^2}} \right| \leq \left| \frac{x^2}{e^{a^2 x^2}} \right| \to 0
\end{equation*}
if $x \to \infty$, by l'H\^opital's rule, hence $\phi(x) \to 0$ if $x \to \infty$.
Now apply Theorem 62.II to conclude that the original integral converges.


\subsection*{Exercise 6.25}

Test the convergence of the following integral
\begin{equation*}
    \int_0^{\infty} e^{-x^2 - \frac{a^2}{x^2}} dx.
\end{equation*}

\subsection*{Exercise 6.25}

Note that $0 \leq e^{\frac{a^2}{x^2}} \leq 1$ for all $0 \leq x < \infty$.
So
\begin{equation*}
    \left| \int_0^{\infty} e^{-x^2 - \frac{a^2}{x^2}} dx \right|
        \leq \int_0^{\infty} \left|e^{-x^2}\right| \left|e^{-\frac{a^2}{x^2}}\right| dx
        \leq \int_0^{\infty} e^{-x^2} dx.
\end{equation*}
Following Example 62.2 we take $\phi(x) = x^2e^{-x^2}$.
Then $\phi(x) \to 0$ if $x \to \infty$, so by Theorem 62.II
\begin{equation*}
    \int_0^{\infty} e^{-x^2} dx = \int_0^{\infty} \frac{\phi(x)}{x^2} dx
\end{equation*}
converges.
And, therefore, the original integral converges as well.


\subsection*{Exercise 6.26}

Test the convergence of the following integral
\begin{equation*}
    \int_0^{\infty} \frac{dx}{\sqrt{x^3 - 1}}.
\end{equation*}

\subsection*{Solution}
Rewrite
\begin{equation*}
    \int_0^{\infty} \frac{dx}{\sqrt{x^3 - 1}}
        = \int_0^{\infty} \frac{1}{x^{\frac{3}{2}}} \frac{x^{\frac{3}{2}}}{(x^3 - 1)^{\frac{1}{2}}} dx.
        = \int_0^{\infty} \frac{\phi(x)}{x^{\frac{3}{2}}} dx.
\end{equation*}
We calculate the limit as $x \to \infty$ for
\begin{equation*}
    \phi(x) = \frac{x^{\frac{3}{2}}}{(x^3 - 1)^{\frac{1}{2}}}
        = \left( \frac{x^3}{x^3 - 1} \right)^{\frac{1}{2}}
        \to 1.
\end{equation*}
Now apply Theorem 62.II to show that the original integral converges.


\subsection*{Exercise 6.27}

Prove the convergence of
\begin{equation*}
    \int_0^{\infty} \sin(x^2) dx.
\end{equation*}

\subsection*{Solution}

Write
\begin{equation*}
    \int_0^{\infty} \sin(x^2) dx
        = \sum_{k = 0}^{\infty} \int_{\sqrt{k \pi}}^{\sqrt{(k + 1) \pi}} \sin(x^2) dx
        = \sum_{k = 0}^{\infty} (-1)^k u_k,
\end{equation*}
where
\begin{equation*}
    u_k = \left| \int_{\sqrt{k \pi}}^{\sqrt{(k + 1)\pi}} \sin(x^2) dx \right| \leq (\sqrt{k+1} - \sqrt{k}) \sqrt{\pi}.
\end{equation*}
Note that $(\sqrt{k+1} - \sqrt{k}) \to 0$ if $k \to \infty$, so $u_k \to 0$ as well.
Also $u_k > u_{k+1}$.
Therefore, we can apply 29.(2) to show that $\sum_{k = 0}^{\infty} (-1)^k u_k$ converges, and hence the integral converges as well.


\subsection*{Exercise 6.28}

Prove the convergence of
\begin{equation*}
    \int_0^{\infty} \frac{e^{-ax} \sin(mx)}{x} dx.
\end{equation*}

\subsection*{Solution}
Rewrite
\begin{equation*}
    \int_0^{\infty} \frac{e^{-ax} \sin(mx)}{x} dx
        = \int_0^{\infty} \frac{\phi(x)}{x^2} dx, \quad \phi(x) = \frac{x \sin(mx)}{e^{ax}}.
\end{equation*}
Note that $|\phi(x)| = xe^{-ax} \to 0$ if $x \to \infty$, so $\phi(x) \to \infty$ as well.
Now apply Theorem 62.II to show that the integral converges.


\subsection*{Exercise 6.29}

Prove that the conditions for differentiability with respect to $\alpha$ under the integral sign is satisfied for
\begin{equation*}
    \int_0^{\infty} e^{-\alpha x} dx.
\end{equation*}

\subsection*{Solution}

We need $\alpha > 0$, otherwise the integral doesn't converge.
Let $f(x, \alpha) = x^{2} e^{-\alpha x}$, such that $f_{\alpha}(x, \alpha) = -x^{3} e^{-\alpha x}$.
Now $f$ and $f_{\alpha}$ are both continuous for $0 < x < \infty$.
Take $\ell > 0$ and write
\begin{equation*}
    F(\alpha) = \int_0^{\infty} f(x, \alpha) dx = \int_0^{\ell} f(x, \alpha) dx + \int_{\ell}^{\infty} f(x, \alpha) dx.
\end{equation*}
For $\ell$ large enough, we have
\begin{equation*}
    f(x, \alpha) = \frac{x^4 e^{-\alpha x}}{x^2} < \frac{1}{x^2},
    \quad \left|f_{\alpha}(x, \alpha)\right| = \left|-\frac{x^5 e^{-\alpha x}}{x^2}\right| < \frac{1}{x^2},
\end{equation*}
when $x > \ell$.
Therefore, $F(\alpha)$ is continuous and differentiable, and the derivative $F'(\alpha)$ can be evaluated by differentiating under the integral sign.


\subsection*{Exercise 6.30}

Prove that the conditions for differentiability with respect to $\alpha$ under the integral sign is satisfied for
\begin{equation*}
    F(\alpha) = \int_0^{\infty} e^{-\alpha x^2} dx.
\end{equation*}

\subsection*{Solution}

We need $\alpha > 0$, otherwise the integral doesn't converge.
Let $f(x, \alpha) = e^{-\alpha x^2}$, such that $f_{\alpha}(x, \alpha) = -x^2 e^{-\alpha x}$.
Note that $f$ and $f_{\alpha}$ are continuous for $0 < x < \infty$.
For large enough $x$ we have
\begin{equation*}
    |f(x, \alpha)| = |e^{-\alpha x^2}| = \left| \frac{x^2 e^{-\alpha x^2}}{x^2} \right| < \frac{1}{x^2},
\end{equation*}
and
\begin{equation*}
    |f_{\alpha}(x, \alpha)| = |-x^2 e^{-\alpha x}| = \left| \frac{-x^4 e^{-\alpha x}}{x^2} \right| < \frac{1}{x^2}.
\end{equation*}
So we may conclude that $F(\alpha)$ is continuous and differentiable, and the derivative $F'(\alpha)$ can be evaluated by differentiating under the integral sign.


\subsection*{Exercise 6.31}

Prove that the conditions for differentiability with respect to $\alpha$ under the integral sign is satisfied for
\begin{equation*}
    F(\alpha) = \int_{0}^{\infty} e^{-bx^2} \cos(\alpha x) dx.
\end{equation*}

\subsection*{Solution}

We need $\alpha > 0$ and $b > 0$, otherwise the integral doesn't converge.
Let $f(x, \alpha) = e^{-bx^2} \cos(\alpha x)$, such that $f_{\alpha}(x, \alpha) = -x e^{-bx^2} \sin(\alpha x)$.
Note that $f$ and $f_{\alpha}$ are continuous for $0 < x < \infty$.
For large enough $x$ we have
\begin{equation*}
    |f(x, \alpha)| = |e^{-b x^2} \cos(\alpha x)| = \left| \frac{x^2 e^{-b x^2} \cos(\alpha x)}{x^2} \right| < \frac{1}{x^2},
\end{equation*}
and
\begin{equation*}
    |f_{\alpha}(x, \alpha)| = |-x e^{-b x} \sin(\alpha x)| = \left| \frac{-x^3 e^{-b x} \sin(\alpha x)}{x^2} \right| < \frac{1}{x^2}.
\end{equation*}
So we may conclude that $F(\alpha)$ is continuous and differentiable, and the derivative $F'(\alpha)$ can be evaluated by differentiating under the integral sign.


\subsection*{Exercise 6.32}

Prove that the conditions for differentiability with respect to $\alpha$ under the integral sign is satisfied for
\begin{equation*}
    F(\alpha) = \int_{0}^{\infty} \frac{dx}{x^2 + \alpha}.
\end{equation*}

\subsection*{Solution}

We solve the integral directly.
For $\alpha \neq 0$,
\begin{equation*}
    F(\alpha) = \int_0^{\infty} \frac{dx}{x^2 + \alpha}
        = \frac{1}{\alpha} \int_0^{\infty} \frac{dx}{\frac{x^2}{\alpha} + 1}
        = \frac{1}{\sqrt{\alpha}} \int_0^{\infty} \frac{du}{u^2 + 1}
        = \left. \frac{1}{\sqrt{\alpha}} \arctan(u) \right|_0^{\infty}
        = \frac{\pi}{2 \sqrt{\alpha}},
\end{equation*}
where we use substitution $u = \frac{x}{\sqrt{\alpha}}$, hence $du = \frac{dx}{\sqrt{\alpha}}$.
So $F$ is continuous and differentiable in $\alpha \neq 0$.


\subsection*{Exercise 6.33}

From
\begin{equation*}
    \int_0^{\infty} e^{-\alpha x} dx = \frac{1}{\alpha},
\end{equation*}
obtain by differentiation
\begin{equation*}
    \int_0^{\infty} x^n e^{-\alpha x} dx = \frac{n!}{\alpha^{n+1}}.
\end{equation*}

\subsection*{Solution}

Define
\begin{equation*}
    I(\alpha) = \int_0^{\infty} e^{-\alpha x} dx
        = \left. \frac{-1}{\alpha} e^{-\alpha x} \right|_0^{\infty}
        = \frac{1}{\alpha}.
\end{equation*}
On the one hand, by differentiating under the integral sign, we have
\begin{equation*}
    I^{(n)}(\alpha)
        = \int_0^{\infty} \frac{d^n}{d\alpha^n} e^{-\alpha x} dx
        = \int_0^{\infty} (-1)^n x^n e^{-\alpha x} dx.
\end{equation*}
On the other hand,
\begin{equation*}
    I^{(n)}(\alpha)
        = \frac{d^n}{d\alpha^n} \frac{1}{\alpha}
        = \frac{(-1)^n n!}{\alpha^{n+1}}.
\end{equation*}
Combining both equations and canceling the $(-1)^n$ term, we get the result.


\subsection*{Exercise 6.34}

From
\begin{equation*}
    \int_0^{\infty} e^{-\alpha x^2} dx = \frac{1}{2} \sqrt{\frac{\pi}{\alpha}},
\end{equation*}
obtain by differentiation
\begin{equation*}
    \int_0^{\infty} x^{2n} e^{-\alpha x^2} = \frac{\sqrt{\pi}}{2} \frac{1 \cdot 3 \cdot ... \cdot (2n - 1)}{2^n} \alpha^{-n - \frac{1}{2}}.
\end{equation*}

\subsection*{Solution}

Define
\begin{equation*}
    I(\alpha) = \int_0^{\infty} e^{-\alpha x^2} dx.
\end{equation*}
Using Example 65.1, we calculate
\begin{equation*}
    I(\alpha) = \int_0^{\infty} e^{-\alpha x^2} dx.
        = \frac{1}{\sqrt{\alpha}} \int_0^{\infty} e^{-y^2} dy
        = \frac{1}{2} \sqrt{\frac{\pi}{\alpha}}.
\end{equation*}
Differentiating under the integral sign gives us
\begin{equation*}
    I^{(n)}(\alpha) = \int_0^{\infty} \frac{d^n}{d\alpha^n} e^{-\alpha x^2} dx
        = \int_0^{\infty} (-1)^n x^{2n} e^{-\alpha x^2} dx.
\end{equation*}
On the other hand,
\begin{equation*}
    I^{(n)}(\alpha) = \frac{d^n}{d\alpha^n} \frac{1}{2} \sqrt{\frac{\pi}{\alpha}}
        = (-1)^n \frac{\sqrt{\pi}}{2} \left(\frac{1 \cdot 3 \cdot ... \cdot (2n - 1)}{2^n} \right) \alpha^{-n + \frac{1}{2}}.
\end{equation*}
Canceling the $(-1)^n$ term in both equations gives us the result.


\subsection*{Exercise 6.35}

From
\begin{equation*}
    \int_0^{\infty} \frac{dx}{x^2 + \alpha} = \frac{\pi}{2 \sqrt{\alpha}},
\end{equation*}
obtain by differentiation
\begin{equation*}
    \int_0^{\infty} \frac{dx}{(x^2 + \alpha)^n} = \frac{\pi}{2} \frac{1 \cdot 3 \cdot 5 \cdot ... \cdot (2n - 1)}{2 \cdot 4 \cdot 6 \cdot ... \cdot (2n)} \alpha^{-(n + \frac{1}{2})}.
\end{equation*}

\subsection*{Solution}

Define
\begin{equation*}
    I(\alpha) = \int_0^{\infty} \frac{dx}{x^2 + \alpha}.
\end{equation*}
With substitution $u = \frac{x}{\sqrt{\alpha}}$ we have
\begin{equation*}
    I(\alpha) = \int_0^{\infty} \frac{dx}{x^2 + \alpha}.
        = \frac{1}{\sqrt{\alpha}} \int_0^{\infty} \frac{dx}{u^2 + 1} du
        = \left. \frac{1}{\sqrt{\alpha}} \arctan(u) \right|_0^{\infty}
        = \frac{\pi}{2 \sqrt{\alpha}}.
\end{equation*}
Differentiating under the integral sign gives us
\begin{equation*}
    I^{(n)}(\alpha) = \int_0^{\infty} \frac{d^n}{dx^n} \frac{dx}{x^2 + 1}
        = (-1)^n n! \int_0^{\infty} \frac{dx}{(x^2 + 1)^n}.
\end{equation*}
On the other hand,
\begin{equation*}
    I^{(n)}(\alpha) = \frac{d^n}{d\alpha^n} \frac{\pi}{2 \sqrt{\alpha}}
        = \frac{\pi}{2} (-1)^n \cdot \frac{1}{2} \cdot \frac{3}{2} \cdot \frac{5}{2} \cdot ... \cdot \frac{(2n-1)}{2} \alpha^{-(n + \frac{1}{2})}.
\end{equation*}
Combining both sides and dividing by $(-1)^n n!$ gives us
\begin{equation*}
    \int_0^{\infty} \frac{dx}{(x^2 + \alpha)^n} = \frac{\pi}{2} \frac{1 \cdot 3 \cdot 5 \cdot ... \cdot (2n - 1)}{2 \cdot 4 \cdot 6 \cdot ... \cdot (2n)} \alpha^{-(n + \frac{1}{2})}.
\end{equation*}


\subsection*{Exercise 6.36}

From
\begin{equation*}
    \int_0^{\infty} e^{-\alpha x} \cos(mx) dx = \frac{\alpha}{\alpha^2 + m^2},
\end{equation*}
obtain by integration
\begin{equation*}
    \int_0^{\infty} \frac{e^{-\alpha x} - e^{-\beta x}}{x \sec(mx)} dx = \frac{1}{2} \log\left( \frac{\beta^2 + m^2}{\alpha^2 + m^2} \right).
\end{equation*}

\subsection*{Solution}

For the first integral calculate note that $\cos(mx) = \mathrm{Re}(e^{-imx})$, so
\begin{equation*}
    \int_0^{\infty} e^{-\alpha x} \cos(mx) dx
        = \mathrm{Re} \left( \int_0^{\infty} e^{-\alpha x} e^{-imx} dx \right)
\end{equation*}
Therefore,
\begin{equation*}
    \int_0^{\infty} e^{-(\alpha + im)x} dx
        = \left. \frac{-1}{\alpha + im} e^{-(\alpha + im)x} \right|_0^{\infty}
        = \frac{1}{\alpha + im}
        = \frac{(\alpha - im)}{(\alpha + im)(\alpha - im)}
        = \frac{\alpha}{\alpha^2 + m^2} - i \frac{m}{\alpha_2 + m^2}.
\end{equation*}
Taking the real part of this equation gives the first result.
For the second integral, write
\begin{equation*}
    \int_0^{\infty} \frac{e^{-\alpha x} - e^{-\beta x}}{x \sec(mx)} dx = I(\alpha) - I(\beta),
\end{equation*}
where
\begin{equation*}
    I(\alpha) = \int_0^{\infty} \frac{e^{-\alpha x}}{x \sec(mx)} dx.
\end{equation*}
Differentiating under the integral sign gives
\begin{equation*}
    I'(\alpha) = -\int_0^{\infty} e^{-\alpha x} \cos(mx) dx = -\frac{\alpha}{\alpha^2 + m^2}.
\end{equation*}
Integrating $I'(\alpha)$, we get
\begin{equation*}
    I(\alpha) = -\frac{1}{2} \log(\alpha^2 + m^2) + C,
\end{equation*}
for some constant $C$.
The constant $C$ is irrelevant, because the constant cancels in the target integral $I(\alpha) - I(\beta)$.
We have
\begin{equation*}
    I(\alpha) - I(\beta) = -\frac{1}{2} \log(\alpha^2 + m^2) + \frac{1}{2} \log(\beta^2 + m^2)
        = \frac{1}{2} \log \left( \frac{\beta^2 + m^2}{\alpha^2 + m^2} \right).
\end{equation*}


\subsection*{Exercise 6.37}

From
\begin{equation*}
    \int_0^{\infty} e^{-\alpha x} \sin(mx) dx = \frac{m}{\alpha^2 + m^2},
\end{equation*}
obtain by integration
\begin{equation*}
    \int_0^{\infty} \frac{e^{-\alpha x} - e^{-\beta x}}{x \csc(mx)} dx = \tan^{-1}\left(\frac{\beta}{m}\right) - \tan^{-1}\left(\frac{\alpha}{m}\right).
\end{equation*}

\subsection*{Solution}

The first integral can be obtained using the same method as in Exercise 6.36, but taking the imaginary part instead of the real part.
For the second integral, write
\begin{equation*}
    \int_0^{\infty} \frac{e^{-\alpha x} - e^{-\beta x}}{x \csc(mx)} dx = I(\alpha) - I(\beta),
\end{equation*}
where
\begin{equation*}
    I(\alpha) = \int_0^{\infty} \frac{e^{-\alpha x}}{x \csc(mx)} dx.
\end{equation*}
Differentiating under the integral sign with respect to $\alpha$ gives
\begin{equation*}
    I'(\alpha) = - \int_0^{\infty} e^{-\alpha x} \sin(mx) dx = \frac{m}{\alpha^2 + m^2}.
\end{equation*}
Integrating $I'(\alpha)$ gives
\begin{equation*}
    I(\alpha) = -\tan^{-1}\left( \frac{\alpha}{m} \right) + C.
\end{equation*}
The constant $C$ is irrelevant, because the constant will cancel in the target integral $I(\alpha) - I(\beta)$.
We have
\begin{equation*}
    I(\alpha) - I(\beta) = \tan^{-1}\left(\frac{\beta}{m}\right) - \tan^{-1}\left(\frac{\alpha}{m}\right).
\end{equation*}


\subsection*{Exercise 6.40}

Investigate the convergence of the following integral
\begin{equation*}
    \int_0^1 \log(x)^n dx.
\end{equation*}

\subsection*{Solution}

Use the substitution $u = -\log(x)$ such that $x = e^{-u}$ and $dx = -e^{-u}du$.
We have
\begin{equation*}
    I = \int_0^1 \log(x)^n dx
        = \int_0^{\infty} (-1)^n u^n e^{-u} du.
\end{equation*}
Define
\begin{equation*}
    J(\alpha) = \int_0^{\infty} e^{-u\alpha} du = \frac{1}{\alpha}.
\end{equation*}
Differentiating $n$ times with respect to $\alpha$ gives on the one hand
\begin{equation*}
    J^{(n)}(\alpha) = \frac{d^n}{d\alpha^n} \frac{1}{\alpha} = \frac{(-1)^n n!}{\alpha^n}.
\end{equation*}
On the other hand, differentiating under the integral sign
\begin{equation*}
    J^{(n)}(\alpha) = \int_0^{\infty} \frac{d^n}{d\alpha^n} e^{-u\alpha} du
        = \int_0^{\infty} (-1)^n u^n e^{-u\alpha} du.
\end{equation*}
From this last integral, we see that $I = J^{(n)}(\alpha = 1) = (-1)^n n! / \alpha^n$.
In particular, this integral converges for all $n \geq 0$.


\subsection*{Exercise 6.41}

Investigate the convergence of the following integral
\begin{equation*}
    \int_0^1 \frac{\log(x)}{1 - x^2} dx.
\end{equation*}

\subsection*{Solution}

Use substitution $u = -\log(x)$, such that $du = -e^{u}dx$, and
\begin{equation*}
    I = \int_0^1 \frac{\log(x)}{1 - x^2} dx
        = \int_0^{\infty} \frac{u e^{-u}}{1 - e^{-2u}} du
        = - \int_0^{\infty} \frac{u e^{u}}{1 - e^{2u}} du.
\end{equation*}
Note that, using l'H\^opital's Theorem,
\begin{equation*}
    \lim_{u \to 0} \frac{u e^{u}}{1 - e^{2u}}
        = \lim_{u \to 0} \frac{e^{u} + u e^{u}}{-2 e^{u}}
        = \lim_{u \to 0} \frac{1 + u}{-2e^u}
        = -\frac{1}{2}.
\end{equation*}
So for $\ell > 0$ we can write
\begin{equation*}
    I = -\int_0^{\ell} \frac{u e^{u}}{1 - e^{2u}} du - \int_{\ell}^{\infty} \frac{\phi(u)}{u^2} du,
        \quad \phi(u) = \frac{u^3 e^{u}}{1 - e^{2u}}.
\end{equation*}
Using l'H\^opital's theorem several times, we see $\lim_{u \to \infty} \phi(u) = 0$.
Therefore we can choose $\ell$ large enough such that $|\phi(u)| < 1$ for all $u > \ell$.
Now we have
\begin{equation*}
    |I| \leq \left| \int_0^{\ell} \frac{u e^u}{1 - e^{2u}} du \right|
            + \left| \int_{\ell}^{\infty} \frac{\phi(u)}{u^2} du \right|
        \leq C_{\ell} + \int_{\ell}^{\infty} \frac{1}{u^2} du
        = C_{\ell} + \left[ -\frac{1}{u} \right]_{\ell}^{\infty}
        = C_{\ell} + \frac{1}{\ell},
\end{equation*}
where $C_{\ell}$ is finite.
Therefore $|I| < \infty$, and integral $I$ exists, hence converges.


\subsection*{Exercise 6.42}

Investigate the convergence of the following integral
\begin{equation*}
    \int_0^1 \frac{x dx}{(1 - x^4)^{\frac{1}{3}}}.
\end{equation*}

\subsection*{Solution}

Use the substitution $y = x^4$, so $dy = 4 x^3 dx = 4 y^{-\frac{3}{4}} dx$, to calculate
\begin{equation*}
    \int_0^1 \frac{x dx}{(1 - x^4)^{\frac{1}{3}}}
        = \int_0^1 \frac{1}{4} y^{-\frac{1}{2}} (1 - y)^{-\frac{1}{3}} dy
        = \frac{1}{4} B(\frac{1}{2}, \frac{2}{3})
        = \frac{1}{4} \frac{\Gamma(\frac{1}{2}) \Gamma(\frac{2}{3})}{\Gamma(\frac{7}{6})}
        = \frac{3}{2} \sqrt{\pi} \frac{\Gamma(\frac{2}{3})}{\Gamma(\frac{1}{6})}.
\end{equation*}
The integral exists, so it must converge.

TODO: find a more elementary argument why the integral converges.


\subsection*{Exercise 6.43}

Investigate the convergence of the following integral
\begin{equation*}
    \int_0^{\infty} \frac{dx}{x^{\frac{2}{3}}(1 + x)}.
\end{equation*}

\subsection*{Solution}

Let $0 < \ell_1 < \ell_2$ such that we can split the integral into three parts
\begin{equation*}
    I = \int_0^{\ell_1} \frac{\phi_1(x)}{x^{\frac{2}{3}}} dx + \int_{\ell_1}^{\ell_2} \frac{dx}{x^{\frac{2}{3}}(1 + x)} + \int_{\ell_2}^{\infty} \frac{\phi_2(x)}{u^{\frac{5}{3}}} du.
\end{equation*}
where
\begin{equation*}
    \phi_1(x) = \frac{1}{1 + x}, \quad \phi_2(x) = \frac{x}{1 + x}.
\end{equation*}
The second integral exists because everything is bounded.
For the first integral note that $\phi_1(x) \to 1$ if $x \to 0$.
So we can choose $\ell_1$ such that
\begin{equation*}
    \int_0^{\ell_1} \frac{\phi_1(x)}{x^{\frac{2}{3}}} dx \leq \int_0^{\ell_1} \frac{1}{x^{\frac{2}{3}}} dx
        = \left. 3 x^{\frac{1}{3}} \right|_0^{\ell_1}
        = 3 \ell_1^{\frac{1}{3}}.
\end{equation*}
For the third integral, note that $\phi_2(x) \to 1$ if $x \to \infty$.
Therefore, we can choose $\ell_2 > \ell_1$ such that
\begin{equation*}
    \int_{\ell_2}^{\infty} \frac{\phi_2(x)}{x^{\frac{5}{3}}} dx
        \leq \int_{\ell_2}^{\infty} \frac{1}{x^{\frac{5}{3}}} dx
        = \left. -\frac{3}{2} x^{-\frac{2}{3}} \right|_{\ell_2}^{\infty}
        = \frac{3}{2} \ell_2^{-\frac{2}{3}}.
\end{equation*}
In particular, each integral in $I$ is bounded, hence $I$ must exist and converges.


\subsection*{Exercise 6.44}

Investigate the convergence of the following integral
\begin{equation*}
    \int_1^{\infty} \frac{dx}{x \sqrt{x^2 - 1}}
\end{equation*}

\subsection*{Solution}

First apply a transformation $x \to x + 1$ such that the integral limits range from $0$ to $\infty$.
Next, use the substitution $u = x^{-1}$ such that $dx = -u^{-2} du$ and the integral limits range from $\infty$ to $0$.
Our target integral becomes
\begin{equation*}
    I = \int_0^{\infty} \frac{dx}{(x + 1)\sqrt{x(x + 2)}}
        = - \int_{\infty}^{0} \frac{u^{-2}}{(1 + u^{-1})u^{-\frac{1}{2}} (u^{-1} + 2)^{\frac{1}{2}}} du
        = \int_0^{\infty} \frac{du}{(1 + u)(1 + 2u)^{\frac{1}{2}}}.
\end{equation*}
Let $\ell > 0$ such that we can split the target integral into
\begin{equation*}
    I = \int_0^{\ell} \frac{du}{(1 + u)(1 + 2u)^{\frac{1}{2}}} + \int_{\ell}^{\infty} \frac{\phi(u) du}{u^{\frac{3}{2}}},
        \quad \phi(u) = \frac{u^{\frac{3}{2}}}{(1 + u)(1 + 2u)^{\frac{1}{2}}}.
\end{equation*}
The first integral is finite.
Our goal is to show that for some $\ell$ the second integral is finite as well.
Note that
\begin{equation*}
    \lim_{u \to \infty} \phi(u)^2 = \lim_{u \to \infty} \frac{u^3}{(1 + u)^2(1 + 2u)}
        = \lim_{u \to \infty} \frac{u^3}{2u^3 + \mathcal{O}(u^2)}
        = \frac{1}{2}.
\end{equation*}
Hence, $\phi(u) \to \frac{1}{2}\sqrt{2}$ if $u \to \infty$.
So, for large $\ell$ we can bound $|\phi(u)| < 1$ for all $u > \ell$.
The second integral can be absolutely bounded by
\begin{equation*}
    \left| \int_{\ell}^{\infty} \frac{\phi(u) du}{u^{\frac{3}{2}}} \right|
        \leq \int_{\ell}^{\infty} \frac{du}{u^{\frac{3}{2}}}
        = \left. -u^{-\frac{1}{2}} \right|_{\ell}^{\infty}
        = \frac{1}{\sqrt{\ell}} < \infty.
\end{equation*}
So the second integral is absolutely bounded, hence $I$ converges.


\subsection*{Exercise 6.45}

Investigate the convergence of the following integral
\begin{equation*}
    \int_0^1 \sqrt{\frac{1 - k^2 x^2}{1 - x^2}} dx.
\end{equation*}

\subsection*{Solution}

For $0 \leq x \leq 1$ we have $|1 - k^2 x^2| < B^2$ for some $B > 0$.
Therefore,
\begin{equation*}
    \int_0^1 \left| \sqrt{\frac{1 - k^2 x^2}{1 - x^2}} dx \right|
        \leq \int_0^1 \frac{B}{\sqrt{1 - x^2}} dx
        = B \left. \arcsin(x) \right|_0^1
        = \frac{B \pi}{2}.
\end{equation*}
So, the integral is absolutely integrable, hence the integral converges.

Note that the target integral is the Elliptic integral of the second kind with parameter $k^2$.
There doesn't exist a closed formula (without using Elliptic functions) for this target integral.
Using the notation from the book, chapter XVI,
\begin{equation*}
    \int_0^1 \sqrt{\frac{1 - k^2 x^2}{1 - x^2}} dx = E(1, k^2).
\end{equation*}


\subsection*{Exercise 6.46}

Evaluate the following integral
\begin{equation*}
    \int_0^{\infty} \frac{\sin(mx)}{x} dx = \begin{cases}
        \frac{\pi}{2} & \text{if } m > 0, \\
        0 & \text{if } m = 0, \\
        -\frac{\pi}{2} & \text{if } m < 0.
    \end{cases}
\end{equation*}

\subsection*{Solution}

If $m = 0$, the integral is clearly zero.
Let $m \neq 0$ and define
\begin{equation*}
    I(\alpha) = \int_0^{\infty} \frac{e^{-\alpha x} \sin(mx)}{x} dx.
\end{equation*}
This integral converges absolutely for all $\alpha \geq 0$, so we may differentiate under the integral sign
\begin{equation*}
    J(\alpha) = \frac{d}{d\alpha} I(\alpha)
        = - \alpha \int_0^{\infty} e^{-\alpha x} \sin(mx) dx.
\end{equation*}
Using partial integration twice, we find
\begin{equation*}
    J(\alpha) = \left. \frac{\cos(mx)e^{-\alpha x}}{m} \right|_0^{\infty} - \frac{\alpha}{m} \int_0^{\infty} \cos(mx) e^{-\alpha x} dx
        = -\frac{1}{m} - \frac{\alpha}{m} \left( \left. \frac{\sin(mx) e^{-\alpha x}}{m} \right|_0^{\infty} + \frac{\alpha}{m} \int_0^{\infty} \sin(mx) e^{-\alpha x} dx \right).
\end{equation*}
So,
\begin{equation*}
    J(\alpha) = - \frac{m}{\alpha^2 + m^2} \quad \rightarrow \quad I(\alpha) = -\arctan\left(\frac{\alpha}{m}\right) + C.
\end{equation*}
If $\alpha \to \infty$, the $I(\alpha) \to 0$.
Therefore, if $m > 0$, then $C = \frac{\pi}{2}$, and if $m < 0$, then $C = -\frac{\pi}{2}$.
Finally, if $\alpha = 0$, we have our target integral, hence $I(0) = -\arctan(0) + C = C$.


\subsection*{Exercise 6.47}

Evaluate the following integral
\begin{equation*}
    \int_0^{\pi} x \log(\sin(x)) dx = -\frac{\pi^2}{2} \log(2).
\end{equation*}

\subsection*{Solution}

Let $I$ be our target integral.
Use the substitution $y = \pi - x$, such that $dy = -dx$, and note that $\sin(\pi - x) = \sin(x)$.
We have
\begin{equation*}
    \begin{split}
        I &= \int_0^{\pi} x \log(\sin(x)) dx \\
            &= - \int_0^{\pi} (\pi - y) \log(\sin(y)) dy \\
            &= \pi \int_0^{\pi} \log(\sin(y)) dy - \int_0^{\pi} y \log(\sin(y)) dy
            = \pi \int_0^{\pi} \log(\sin(y)) dy - I.
    \end{split}
\end{equation*}
So we have $I = \frac{\pi}{2} J$, where
\begin{equation*}
    J = \int_{0}^{\pi} \log(\sin(x)) dx.
\end{equation*}
Note that
\begin{equation*}
    \int_0^{\pi} \log(\sin(x)) dx = 2 \int_0^{\frac{\pi}{2}} \log(\sin(x)) dx
        = 2 \int_0^{\frac{\pi}{2}} \log(\cos(y)) dy,
\end{equation*}
where we used the substitution $y = \frac{\pi}{2} - x$ in the last equation.
Therefore,
\begin{equation*}
    \begin{split}
        J &= \int_0^{\frac{\pi}{2}} \log(\sin(x)) dx + \int_0^{\frac{\pi}{2}} \log(\cos(x)) dx \\
            &= \int_0^{\frac{\pi}{2}} \log(\sin(x) \cos(x)) dx \\
            &= \int_0^{\frac{\pi}{2}} \log\left(\frac{\sin(2x)}{2}\right) dx \\
            &= \int_0^{\frac{\pi}{2}} \log(\sin(2x)) dx - \int_0^{\frac{\pi}{2}} \log(2) dx \\
            &= \frac{1}{2} \int_{0}^{\pi} \log(\sin(y)) dy - \frac{\pi}{2} \log(2)
            = \frac{1}{2} J - \frac{\pi}{2} \log(2),
    \end{split}
\end{equation*}
where we use the substitution $y = 2x$ in the one but last equation.
So $J = -\pi \log(2)$, and the target integral $I = -\frac{\pi}{2} \log(2)$.


\subsection*{Exercise 6.48}

Evaluate the following integral
\begin{equation*}
    \int_0^{\infty} e^{-\alpha^2 x^2} dx = \frac{\sqrt{\pi}}{2 \alpha}.
\end{equation*}

\subsection*{Solution}

Make the substitution $u = \alpha x$, then
\begin{equation*}
    I = \frac{1}{\alpha} \int_0^{\infty} e^{-u^2} du
        = \frac{1}{\alpha} \int_0^{\infty} e^{-v^2} dv.
\end{equation*}
So, considering $I^2$ as a double integral and using polar coordinates
\begin{equation*}
    I^2 = \frac{1}{\alpha^2} \int_0^{\infty} \int_0^{\infty} e^{-(u^2 + v^2)} dv du
        = \frac{1}{\alpha^2} \int_0^{\infty} \int_0^{\frac{\pi}{2}} e^{-r^2} r d\theta dr
        = \frac{\pi}{4 \alpha^2}.
\end{equation*}
Taking the positive square root of $I^2$ gives the answer.


\subsection*{Exercise 6.49}

Evaluate the following integral
\begin{equation*}
    \int_0^1 \frac{dx}{\sqrt{\log\left(\frac{1}{x}\right)}} = \sqrt{\pi}.
\end{equation*}

\subsection*{Solution}

Take $u^2 = \log(1/x)$ such that $x = e^{-u^2}$ and $dx = -2ue^{-u^2}du$.
We have
\begin{equation*}
    \int_0^1 \frac{dx}{\sqrt{\log\left(\frac{1}{x}\right)}}
        = - \int_{\infty}^0 2 u^{-1} u e^{-u^2} du
        = 2 \int_0^{\infty} e^{-u^2} du
        = \sqrt{\pi}.
\end{equation*}


\subsection*{Exercise 6.50}

Evaluate the following integral
\begin{equation*}
    \int_0^{\infty} \frac{\sin(x) \cos(mx)}{x} dx = \begin{cases}
        0 & \text{if } m < -1 \text{ or } 1 < m, \\
        \frac{\pi}{4} & \text{if } m = -1 \text{ or } m = 1, \\
        \frac{\pi}{2} & \text{if } -1 < m < 1.
    \end{cases}
\end{equation*}

\subsection*{Solution}

Use the trigonometric identity $2 \sin(x) \cos(x) = \sin(x + y) + \sin(x - y)$, such that
\begin{equation*}
    \int_0^{\infty} \frac{\sin(x) \cos(mx)}{x} dx
        = \frac{1}{2} \int_0^{\infty} \frac{\sin((1 + m)x)}{x} dx + \frac{1}{2} \int_0^{\infty} \frac{\sin((1 - m)x)}{x} dx.
\end{equation*}
Now use exercise 6.46 to evaluate both integrals.
Working out all the specific cases (5 in total) will give the result.


\subsection*{Exercise 6.51}

Evaluate the following integral
\begin{equation*}
    \int_0^{\infty} e^{-x^2 - \frac{\alpha^2}{x^2}} dx = e^{-2\alpha} \frac{\sqrt{\pi}}{2}.
\end{equation*}


\subsection*{Exercise 6.51}

Let
\begin{equation*}
    I(\alpha) = \int_0^{\infty} e^{-x^2 - \frac{\alpha^2}{x^2}} dx.
\end{equation*}
We have
\begin{equation*}
    \frac{dI}{d\alpha}
        = -2 \int_0^{\infty} \frac{\alpha}{x^2} e^{-x^2 - \frac{\alpha^2}{x^2}} dx
        = -2 \int_{\infty}^{0} -e^{-\frac{\alpha^2}{u^2} - u^2} du
        = -2 \int_0^{\infty} e^{-u^2 - \frac{\alpha^2}{u^2}} du
        = -2 I,
\end{equation*}
where we use the substitution $u = \frac{\alpha}{x}$ in the second equality.
Solving the differential equation $\frac{dI}{d\alpha} = -2I$ gives,
\begin{equation*}
    \frac{dI}{I} = -2 d\alpha
        \quad \leftrightarrow \quad \log(I) = -2\alpha + C
        \quad \leftrightarrow \quad I(\alpha) = e^{-2\alpha} C',
\end{equation*}
where $C$ and $C'$ are some constants.
To find the constant $C'$ note that $C = I(\alpha = 0) = \frac{\sqrt{\pi}}{2}$ is a well-know integral (e.g., see Section 65, Example 1).
Hence $I = e^{-2\alpha} \frac{\sqrt{\pi}}{2}$.


\subsection*{Exercise 6.52}

Evaluate the following integral
\begin{equation*}
    \int_0^{\infty} \frac{e^{-a x} \sin(bx)}{x} dx = \tan^{-1} \left(\frac{b}{a}\right).
\end{equation*}

\subsection*{Solution}
Note that we need $a > 0$, since otherwise the integral doesn't converge.
Also, if $\beta = 0$, the integral is zero, so assume $\beta \neq 0$.
Let
\begin{equation*}
    I(a) = \int_0^{\infty} \frac{e^{-a x} \sin(bx)}{x} dx,
\end{equation*}
such that
\begin{equation*}
    I'(a) = - \int_0^{\infty} e^{-ax} \sin(b x) dx.
\end{equation*}
Using partial integration twice
\begin{equation*}
    I'(a) = -\left( \frac{1}{b} + \frac{a^2}{b^2} \int_0^{\infty} e^{-ax} \sin(bx) dx\right)
        = -\frac{1}{b} - \frac{a^2}{b^2} I'(a).
\end{equation*}
Therefore,
\begin{equation*}
    I'(a) = -\frac{1}{b} \frac{1}{1 + \frac{a^2}{b^2}}
        \quad \rightarrow \quad I(a) = -\tan^{-1}\left(\frac{a}{b}\right) + C.
\end{equation*}
Note that $\tan^{-1}(a/b) \to \frac{\pi}{2}$ and $I(a) \to 0$ as $a \to \infty$ and $\beta > 0$, so $C = \frac{\pi}{2}$.
If $\beta < 0$, then $\tan^{-1}(a/b) \to -\frac{\pi}{2}$ and $C = -\frac{\pi}{2}$.
I think the original answer in the book is wrong.
\TODO{Check my results}


\subsection*{Exercise 6.53}

Evaluate the following integrals
\begin{equation*}
    \int_0^{\infty} \frac{\cos(x)}{\sqrt{x}} dx = \int_0^{\infty} \frac{\sin(x)}{\sqrt{x}} dx = \sqrt{\frac{\pi}{2}}.
\end{equation*}

\subsection*{Solution}

Substitute $u = \sqrt{x}$, so that $dx = 2 du$.
We have
\begin{equation*}
    I_1 = 2 \int_0^{\infty} \cos(u^2) du, \quad I_2 = 2 \int_0^{\infty} \sin(u^2) du.
\end{equation*}
Note that $e^{-iu^2} = \cos(u^2) - i \sin(u^2)$, so
\begin{equation*}
    \int_0^{\infty} \cos(u^2) du - i \int_0^{\infty} \sin(u^2) du
        = \int_0^{\infty} e^{-iu^2} du
        = \frac{\sqrt{\pi}}{2 \sqrt{i}}
        = \frac{\sqrt{\pi}}{2} e^{- i \frac{\pi}{4}}
        = \frac{\sqrt{\pi}}{2} \left(\cos\left(-\frac{\pi}{4}\right) - i \sin\left(-\frac{\pi}{4}\right)\right)
        = \frac{\sqrt{2 \pi}}{4} - i \frac{\sqrt{2 \pi}}{4}.
\end{equation*}
Separating the real and imaginary part we have
\begin{equation*}
    I_1 = I_2 = 2 \frac{\sqrt{2 \pi}}{4} = \sqrt{\frac{\pi}{2}}.
\end{equation*}

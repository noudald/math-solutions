\section*{Introduction}

Over a time span of approximately nine months between 2022 and 2023, I engaged in self-study of All of Statistics from Larry Wasserman.
Due to time constrains, I only devoted half an hour to an hour per day to studying.
I believe that it is crucial to complete most of the exercises, as the book is consise and many details are left to the reader.
In this document, you can find my solutions to most of the exercises of the book.

The first two parts of the book are well-organized and contain interesting exercises.
In my opinion, the quality of the last part of the book inferior to the first two parts.
This issue is particularly evident in the final section of the book.
There are several issues with some of the Theorems and Algorithms presented in these chapters.
Not all exercises are well-defined, which is due to missing details in the text as it tries to be a consise summary of advanced statistics.
Another issue is that not all algorithms can be directly implemented due to floating point overflows.
As a result of these issues, you may notice that my solutions become more sloppy in the last part of the book.
I would also like to note that I typically write down the solution few details.
The solutions are written in a way that I can understand them.
While many mathematicians prefer more detail, I'm not one of them.

I admit that I did not fully comprehend Chapter 16, Causual Inference.
I'm unsure if I will revisit this chapter and redo the exercises.
Lastly, I'll acknowledge that I did look to other solutions as well from time to time.
If I was stuck, I searched for (partial) solutions online.

Although this book has its issues, my overall experience was very positive.
All of Statistics by Larry Wasserman is precisely what it claims to be on the cover: a concise course in statistical inference.
A revised version of the book that addresses these mistakes would be beneficial.
Nonetheless, it's a good book if you want to quickly become familiar with the topics, or to be used as a reference.

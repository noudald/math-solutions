\section*{Chapter 7 - Operators on Inner Product Spaces}

\subsection*{Section 7A - Self-Adjoint and Normal Operators}

\subsubsection*{Exercise 1}

Suppose $n$ is a positive integer.
Define $T \in \mathcal{L}(\mathbb{F}^n)$ by $T(x_1, x_2, ..., x_n) = (0, x_1, ..., x_{n-1})$.
Find a formula for $T^*(x_1, x_2, ..., x_n)$.

\subsubsection*{Solution}

Take $T^*(y_1, y_2, ..., y_n) = (y_2, y_3, ..., y_n, 0)$, then
\begin{equation*}
    \begin{split}
        \langle T(x_1, x_2, ..., x_n), (y_1, y_2, ..., y_n) \rangle
            &= \langle (0, x_1, x_2, ..., x_{n-1}), (y_1, y_2, ..., y_n) \rangle \\
            &= x_1 y_2 + x_2 y_3 + ... + x_{n-1} y_n \\
            &= \langle (x_1, x_2, ..., x_n), (y_2, y_3, ..., y_{n-1}, 0) \rangle \\
            &= \langle (x_1, x_2, ..., x_n), T^*(y_1, y_2, ..., y_n) \rangle.
    \end{split}
\end{equation*}


\subsubsection*{Exercise 2}

Suppose $T \in \mathcal{L}(V, W)$.
Prove that the following are equivalent
\begin{enumerate}
\item $T = 0$,
\item $T^* = 0$,
\item $T^* T = 0$,
\item $T T^* = 0$.
\end{enumerate}

\subsubsection*{Solution}

Note that $\langle Tv, w\rangle = \langle v, T^*w \rangle$, for all $v, w$.
In particular $\langle Tv, Tv \rangle = \langle v, T^*Tv \rangle = \langle T^*T v, v \rangle$ and $\langle T^*w, T^*w\rangle = \langle T T^* w, w \rangle$ for all $v \in V$ and $w \in W$.
If $T = 0$, then $T^*T = 0$ and $TT^* = 0$.
Conversely if $T^*T = 0$ or $TT^* = 0$, then, using the first equation, $\langle Tv, Tv \rangle = 0$, hence $Tv = 0$ for all $v \in V$.
Therefore, $T = 0$.
If $T^* = 0$, then $T^*T = 0$ and $TT^* = 0$.
Conversely if $T^*T = 0$ or $TT^* = 0$, then, using the second equation, $\langle T^*v, T^*v \rangle = 0$, hence $T^*v = 0$ for all $v \in V$.
Therefore, $T^* = 0$.


\subsubsection*{Exercise 3}

Suppose $T \in \mathcal{L}(V)$ and $\lambda \in \mathbb{F}$.
Prove that $\lambda$ is an eigenvalue of $T$ if and only if $\overline{\lambda}$ is an eigenvalue of $T^*$.


\subsubsection*{Solution}

Note that from the previous exercise $T - \lambda I = 0$ if and only if $T^* - \overline{\lambda} I = (T - \lambda I)^* = 0$.
Which gives the result.


\subsubsection*{Exercise 4}

Suppose $T \in \mathcal{L}(V)$ and $U$ a subspace of $V$.
Show that $U$ is invariant under $T$ if and only if $U^{\perp}$ is invariant under $T^*$.

\subsubsection*{Solution}

\begin{itemize}
    \item[$\rightarrow)$] Let $v \in U^{\perp}$, so for all $u \in U$ we have $\langle u, v \rangle = 0$.
        We have, for all $v$, $\langle u, T^*v\rangle = \langle Tu, v\rangle = 0$, because $Tu \in U$, so $T^*u \in U^{\perp}$.
    \item[$\leftarrow)$] Let $u \in U$.
        For all $v \in U^{\perp}$, we have $\langle Tu, v \rangle = \langle u, T^*v \rangle = 0$, because $T^*v \in U^{\perp}$.
        Therefore $Tu \in (U^{\perp})^{\perp} = U$.
\end{itemize}

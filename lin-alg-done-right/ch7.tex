\section*{Chapter 7 - Operators on Inner Product Spaces}

\subsection*{Section 7A - Self-Adjoint and Normal Operators}

\subsubsection*{Exercise 1}

Suppose $n$ is a positive integer.
Define $T \in \mathcal{L}(\mathbb{F}^n)$ by $T(x_1, x_2, ..., x_n) = (0, x_1, ..., x_{n-1})$.
Find a formula for $T^*(x_1, x_2, ..., x_n)$.

\subsubsection*{Solution}

Take $T^*(y_1, y_2, ..., y_n) = (y_2, y_3, ..., y_n, 0)$, then
\begin{equation*}
    \begin{split}
        \langle T(x_1, x_2, ..., x_n), (y_1, y_2, ..., y_n) \rangle
            &= \langle (0, x_1, x_2, ..., x_{n-1}), (y_1, y_2, ..., y_n) \rangle \\
            &= x_1 y_2 + x_2 y_3 + ... + x_{n-1} y_n \\
            &= \langle (x_1, x_2, ..., x_n), (y_2, y_3, ..., y_{n-1}, 0) \rangle \\
            &= \langle (x_1, x_2, ..., x_n), T^*(y_1, y_2, ..., y_n) \rangle.
    \end{split}
\end{equation*}


\subsubsection*{Exercise 2}

Suppose $T \in \mathcal{L}(V, W)$.
Prove that the following are equivalent
\begin{enumerate}
\item $T = 0$,
\item $T^* = 0$,
\item $T^* T = 0$,
\item $T T^* = 0$.
\end{enumerate}

\subsubsection*{Solution}

Note that $\langle Tv, w\rangle = \langle v, T^*w \rangle$, for all $v, w$.
In particular $\langle Tv, Tv \rangle = \langle v, T^*Tv \rangle = \langle T^*T v, v \rangle$ and $\langle T^*w, T^*w\rangle = \langle T T^* w, w \rangle$ for all $v \in V$ and $w \in W$.
If $T = 0$, then $T^*T = 0$ and $TT^* = 0$.
Conversely if $T^*T = 0$ or $TT^* = 0$, then, using the first equation, $\langle Tv, Tv \rangle = 0$, hence $Tv = 0$ for all $v \in V$.
Therefore, $T = 0$.
If $T^* = 0$, then $T^*T = 0$ and $TT^* = 0$.
Conversely if $T^*T = 0$ or $TT^* = 0$, then, using the second equation, $\langle T^*v, T^*v \rangle = 0$, hence $T^*v = 0$ for all $v \in V$.
Therefore, $T^* = 0$.


\subsubsection*{Exercise 3}

Suppose $T \in \mathcal{L}(V)$ and $\lambda \in \mathbb{F}$.
Prove that $\lambda$ is an eigenvalue of $T$ if and only if $\overline{\lambda}$ is an eigenvalue of $T^*$.


\subsubsection*{Solution}

Note that from the previous exercise $T - \lambda I = 0$ if and only if $T^* - \overline{\lambda} I = (T - \lambda I)^* = 0$.
Which gives the result.


\subsubsection*{Exercise 4}

Suppose $T \in \mathcal{L}(V)$ and $U$ a subspace of $V$.
Show that $U$ is invariant under $T$ if and only if $U^{\perp}$ is invariant under $T^*$.

\subsubsection*{Solution}

\begin{itemize}
    \item[$\rightarrow)$] Let $v \in U^{\perp}$, so for all $u \in U$ we have $\langle u, v \rangle = 0$.
        We have, for all $v$, $\langle u, T^*v\rangle = \langle Tu, v\rangle = 0$, because $Tu \in U$, so $T^*u \in U^{\perp}$.
    \item[$\leftarrow)$] Let $u \in U$.
        For all $v \in U^{\perp}$, we have $\langle Tu, v \rangle = \langle u, T^*v \rangle = 0$, because $T^*v \in U^{\perp}$.
        Therefore $Tu \in (U^{\perp})^{\perp} = U$.
\end{itemize}


\subsubsection*{Exercise 5}

Suppose $T \in \mathcal{L}(V, W)$.
Suppose $e_1, e_2, ..., e_n$ is an orthonormal basis of $V$ and $f_1, f_2, ..., f_m$ is an orthonormal basis of $W$.
Prove that
\begin{equation*}
    ||Te_1||^2 + ... + ||Te_n||^2 = ||T^*f_1||^2 + ... + ||T^*f_m||^2.
\end{equation*}

\subsubsection*{Solution}

Because $\{e_i\}$ and $\{f_j\}$ are orthonormal basis we can write
\begin{equation*}
    Te_i = \langle Te_i, f_1 \rangle f_1 + ... + \langle Te_i, f_m \rangle f_m = \sum_{j=1}^m \langle Te_i, f_j \rangle f_j,
\end{equation*}
and
\begin{equation*}
    T^*f_j = \langle T^*f_j, e_1 \rangle e_1 + ... + \langle T^*f_j, e_n \rangle = \sum_{i=1}^n \langle T^*f_j, e_i \rangle e_i.
\end{equation*}
We have
\begin{equation*}
    \begin{split}
        ||Te_1||^2 + ... + ||Te_n||^2
            &= \sum_{i=1}^n \langle Te_i, Te_i \rangle \\
            &= \sum_{i=1}^n \langle \sum_{j=1}^m \langle Te_i, f_j \rangle f_j, Te_i \rangle \\
            &= \sum_{i=1}^n \sum_{j=1}^m \langle Te_i, f_j \rangle \langle f_j, Te_i \rangle \\
            &= \sum_{j=1}^m \sum_{i=1}^n \langle e_i, T^*f_j \rangle \langle T^*f_j, e_i \rangle \\
            &= \sum_{j=1}^m \sum_{i=1}^n \langle T^*f_j, e_i \rangle \langle e_i, T^*f_j \rangle \\
            &= \sum_{j=1}^m \langle \sum_{i=1}^n \langle T^*f_j, e_i \rangle e_i, T^*f_j \rangle \\
            &= \sum_{j=1}^m \langle T^*f_j, T^*f_j \rangle
            = ||T^*f_1||^2 + ... + ||T^*f_m||^2.
    \end{split}
\end{equation*}


\subsubsection*{Exercise 6}

Suppose $T \in \mathcal{L}(V, W)$.
Prove that
\begin{itemize}
\item[(a)] $T$ is injective if and only if $T^*$ is surjective.
\item[(b)] $T$ is surjective if and only if $T$ is injective.
\end{itemize}

\subsubsection*{Solution}

Follows from 7.6.
$T$ is injective if and only if $\nnull(T) = \{0\} = \range(T^*)^{\perp}$ if and only if $\range(T^*) = V$.
$T$ is surjective if and only if $\range(T) = V = \nnull(T^*)^{\perp}$ if and only if $\nnull(T^*) = \{0\}$.


\subsubsection*{Exercise 7}

Prove that if $T \in \mathcal{L}(V, W)$, then
\begin{itemize}
\item[(a)] $\dim(\nnull(T^*)) = \dim(\nnull(T)) + \dim(W) - \dim(V)$.
\item[(b)] $\dim(\range(T^*)) = \dim(\range(T))$.
\end{itemize}

\subsubsection*{Solution}

By 7.6 we have $\nnull(T^*) = \range(T)^{\perp}$, so, using 6.51, $\dim(\nnull(T^*)) = \dim(\range(T)^{\perp}) = \dim(W) - \dim(\range(T))$.
The fundamental theorem of linear maps, 3.21, gives $\dim(\range(T)) = \dim(V) - \dim(\nnull(T))$.
We get (a) by combining the previous results
\begin{equation*}
    \begin{split}
        \dim(\nnull(T^*)) &= \dim(\range(T)^{\perp}) \\
            &= \dim(W) - \dim(\range(T))
            = \dim(\nnull(T)) + \dim(W) - \dim(V).
    \end{split}
\end{equation*}
For (b) we have something similar.
We have
\begin{equation*}
    \begin{split}
        \dim(\range(T^*)) &= \dim(\nnull(T)^{\perp}) \\
            &= \dim(V) - \dim(\nnull(T)) \\
            &= \dim(V) - \dim(V) + \dim(\range(T))
            = \dim(\range(T)).
    \end{split}
\end{equation*}


\subsubsection*{Exercise 8}

Suppose $A$ is an $m$-by-$n$ matrix with entries in $\mathbb{F}$.
Use Exercise 7A.b to prove that the row rank of $A$ equals the column rank of $A$.

\subsubsection*{Solution}

Let $T \in \mathcal{L}(V, W)$ such that $\mathcal{M}(T) = A$.
We have
\begin{equation*}
    \text{column rank } A
        = \dim(\range(T))
        = \dim(\range(T^*))
        = \text{column rank } A^*
        = \text{row rank } A.
\end{equation*}
Exercise 7A.b is used in the second equality, theorem 7.9 is used in the last equality.
